\documentclass{uebungsblatt}

\DeclareMathOperator{\im}{im}
\DeclareMathOperator{\Idl}{Idl}
\DeclareMathOperator{\Rad}{Rad}
\DeclareMathOperator{\Fix}{Fix}

\begin{document}
\maketitle{Universität Augsburg}{Lehrstuhl für Algebra und Zahlentheorie}{Wintersemester 2021/22}{Algebraic Geometry I}{Dr.~Marco Ramponi}{Lukas Stoll, M.Sc.}

Eine \emph{teilweise geordnete Menge} ist eine Menge $A$ zusammen mit einer reflexiven, transitiven und antisymmetrischen Relation $≤$.
Eine Abbildung $f:A → B$ zwischen teilweise geordneten Mengen heißt \emph{ordnungserhaltend} falls $a ≤_A a'$ impliziert, dass $f(a) ≤_B f(a')$.

\begin{exercise}[Galoisverbindungen]
  Eine \emph{Galoisverbindung} zwischen zwei teilweise geordneten Mengen ist ein Paar ordnungserhaltender Abbildungen $g:(B,≤_B) ⇆ (A,≤_A):f$, welches folgende \emph{Adjunktionseigenschaft} für alle $a ∈ A$, $b ∈ B$ erfüllt:
  $$
  f(a) ≤_B b ⇔ a ≤_A g(b)
  $$
  \begin{enumerate}
    \item Seien $X$, $Y$ Mengen und $⊥ ⊆ X × Y$ eine Relation von $X$ nach $Y$.
      Zeige, dass die folgenden Abbildungen eine Galoisverbindung zwischen den Potenzmengen von $X$ und $Y$ bilden.
      \begin{align*}
        𝒱_R : (𝒫(X),⊇) &\leftrightarrows (𝒫(Y),⊆) : ℐ_R\\
        S & \longmapsto \{y ∈ Y \mid ∀s ∈ S:s \perp y\}\\
        \{x ∈ X \mid ∀t ∈ T:x \perp t\} & \longmapsfrom T
      \end{align*}
      {\scriptsize Beachte, dass hier $≤_A$ die Teilmengenrelation $⊆$ und $≤_B$ die umgekehrte Teilmengenrelation $⊇$ ist!
      Eine ordungserhaltende Abbildung würde man hier also eher inklusions\emph{umkehrend} nennen.}
    \item Sei $K$ ein Körper.
      Folgere die Existenz einer Galoisverbindung
      $$
      𝒱:(𝒫(K[x_1,\dots,x_n]),⊇) ⇆ (𝒫(𝔸^n_K),⊆):ℐ
      $$
      zwischen den Potenzmengen des Polynomrings $K[x_1,\dots,x_n]$ und des affinen Raums $𝔸^n_K$, sodass $𝒱(S)$ die Verschwindungsmenge von $S⊆K[x_1,\dots,x_n]$ und $ℐ(T)$ das Verschwindungsideal von $T⊆𝔸^n_K$ ist.
    \item Bezeichne $\Idl(R)$ die Menge der Ideale eines kommutativen Rings $R$.
      Zeige, dass $ℐ(T)$ für jedes $T⊆𝔸^n_K$ ein Ideal ist und folgere, dass obige Galoisverbindung zu einer Galoisverbindung $𝒱:(\Idl(K[x_1,\dots,x_n]),⊇) ⇆ (𝒫(𝔸^n_K),⊆):ℐ$ absteigt.
    \item Bezeichne $\Rad(R)$ die Menge der radikalen Ideale eines kommutativen Rings $R$.
    Zeige, dass $ℐ(T)$ für jedes $T⊆𝔸^n_K$ radikal ist und folgere, dass obige Galoisverbindung zu einer Galoisverbindung $𝒱:(\Rad(K[x_1,\dots,x_n]),⊇) ⇆ (𝒫(𝔸^n_K),⊆):ℐ$ absteigt.\\
      {\scriptsize Ein Ideal $I⊆R$ heißt \emph{radikal}, falls für jedes $f ∈ R$ die Existenz eines $n ∈ ℕ$ mit $f^n ∈ I$ schon $f ∈ I$ impliziert.}
    %\item Finde eine Galoisverbindung zwischen den Zwischenkörpern $E$ einer Galoiserweiterung $L/K$ und den Untergruppen der Galoisgruppe $\mathrm{Gal}(L/K)$.
  \end{enumerate}
\end{exercise}

\begin{exercise}[Hüllenoperatoren]
  Ein \emph{Hüllenoperator} auf einer teilweise geordneten Menge $A$ ist eine ordnungserhaltende Abbildung $◯:A → A$, welche \emph{idempotent} und \emph{extensiv} ist.
  Das heißt $◯(◯(a))=◯(a)$ und $a ≤ ◯(a)$ für alle $a ∈ A$.

  Sei im Folgenden $R$ ein kommutativer Ring und $K$ ein Körper.
  \begin{enumerate}
    \item Zeige, dass die Abbildung $(\_):𝒫(R) → 𝒫(R)$, welche einer Teilmenge $S⊆R$ das von ihr erzeugte Ideal $(S)\coloneqq \{r_1s_1 + \dots + r_ns_n \mid n ∈ ℕ,\, r_i ∈ R,\, s_i ∈ S\}$ zuordnet, ein Hüllenoperator auf $(𝒫(R),⊆)$ ist.
    \item Zeige, dass $𝒱((S))=𝒱(S)$ für jede Teilmenge $S⊆K[x_1,\dots,x_n]$.
    \item Zeige, dass die Abbildung $\sqrt{-}:\Idl(R) → \Idl(R)$, welche einem Ideal $I$ sein Radikal $\sqrt{I}\coloneqq\{r ∈ R \mid ∃n ∈ ℕ : r^n ∈ I\}$ zuordnet, ein Hüllenoperator auf $(\Idl(R),⊆)$ ist.
    \item Zeige, dass $𝒱(\sqrt{I})=𝒱(I)$ für jedes Ideal $I⊆K[x_1,\dots,x_n]$.
  \end{enumerate}
\end{exercise}

\begin{exercise}[Galoiskorrespondenzen]

  Sei $g:B ⇆ A:f$ eine Galoisverbindung.
  \begin{enumerate}
    \item Zeige, dass $g∘f$ ein Hüllenoperator auf $(A,≤_A)$ ist.
    \item Zeige, dass $f∘g$ ein Hüllenoperator auf $(B,≥_B)$ ist. {\scriptsize Beachte die umgekehrte Ordnung.}
    \item Eine Galoisverbindung deren Abbildungen zueinander invers sind, heißt \emph{Galoiskorrespondenz}.
      Zeige, dass die gegebene Galoisverbindung eine Galoiskorrespondenz zwischen den Fixpunktmengen ihrer assoziierten Hüllenoperatoren induziert.\\
      \begin{equation*}
        \begin{tikzcd}
          (\Fix(f∘g),≤_B)
          \arrow[r,yshift=-2.5,"g"']
          & (\Fix(g∘f),≤_A)
          \arrow[l,yshift=2.5,"f"', "\sim" yshift=0.5pt]
        \end{tikzcd}
      \end{equation*}
      {\scriptsize Zur Erinnerung: Ein Fixpunkt einer Selbstabbildung $s:X → X$ ist ein Element $x ∈ X$ mit $s(x)=x$.}
  \end{enumerate}
  Sei $K$ ein algebraisch abgeschlossener Körper.
  \begin{enumerate}[start=4]
    \item Verwende Hilberts Nullstellensatz um die Existenz einer Galoiskorrespondenz zwischen den Radikalidealen von $K[x_1,\dots,x_n]$ und den affinen Varietäten in $𝔸^n_K$ zu zeigen.
    \item Zeige, dass diese Galoiskorrespondenz einschränkt zu einer Galoiskorrespondenz zwischen den maximalen Idealen von $K[x_1,\dots,x_n]$ und den Punkten in $𝔸^n_K$.
  \end{enumerate}
\end{exercise}

%\begin{exercise}[Suprema, Infima und Idealoperationen]
%  Sei $(A,≤)$ eine teilweise geordnete Menge und $M⊆A$.
%  Ein \emph{Supremum} von $M$ ist ein Element $a ∈ A$, welches folgende Eigenschaften erfüllt:
%  \begin{align*}
%    &∀m ∈ M:m ≤ a\\
%    &∀a' ∈ A: (∀m ∈ M:m ≤ a') ⇒ a ≤ a'
%  \end{align*}
%  Dual dazu ist ein \emph{Infimum} von $M$ definiert als Supremum von $M$ aufgefasst als Teilmenge der umgekehrt geordneten Menge $(A,≥)$.
%  \begin{enumerate}
%    \item Zeige, dass Supremum und Infimum eindeutig sind, falls sie existieren.
%      In diesem Fall schreiben wir $⋁ M$ für das Supremum und $⋀ M$ für das Infimum.
%    \item Sei $◯:A → A$ ein Hüllenoperator auf einer teilweise geordnete Menge $(A,≤)$.
%      \begin{enumerate}
%        \item Zeige, dass $◯(a) = ⋀ \{x ∈ \Fix(◯) \mid a ≤ x\}$ für alle $a ∈ A$.
%      \end{enumerate}
%      Wir fassen die Fixpunktmenge von $◯$ nun als teilweise geordnete Teilmenge $(\Fix(◯),≤)$ von $(A,≤)$ auf.
%      Sei $M⊆\Fix(◯)$ und notiere mit $⋀ M$ bzw. $⋁ M$ das Infimum bzw. Supremum von $M$ \emph{in $A$}.
%      Zeige:
%      \begin{enumerate}[start=2]
%        \item Das Infimum von $M$ in $\Fix(M)$ ist gegeben als
%        \item Das Supremum einer Teilmenge $M⊆\Fix(◯)$ in $(\Fix(◯),≤\vert_{\Fix(◯})$ ist gegeben durch $◯(⋁ M)$, wobei $⋁ M$ das Supremum von $M$ in $A$ ist.
%      \end{enumerate}
%
%    \item Berechne das Supremum und Infimum
%      \begin{enumerate}
%        \item einer Menge $M⊆𝒫(X)$ von Teilmengen einer Menge $X$,
%        \item einer Menge $M⊆\Idl(R)$ von Idealen eines kommutativen Rings $R$,
%        \item einer Menge $M⊆\Rad(R)$ von Radikalen eines kommutativen Rings $R$.
%      \end{enumerate}
%    \item Sei $g:B ⇆ A:f$ eine Galoisverbindung.
%      Zeige, dass $g$ Infima und $f$ Suprema erhält.
%      Das heißt für jede Teilmenge $M⊆B$ und $N⊆A$ ist
%      \begin{align*}
%        g \left( ⋀ M \right) = ⋀ g(M),\\
%        f \left( ⋁ M \right) = ⋁ f(M).
%      \end{align*}
%      Folgere die nachfolgenden Regeln für die Verschwindungsmenge einer Familie von Idealen $(𝖆_i)_{i ∈ I}$ des Polynomrings $K[x_1,\dots,x_n]$ über einem Körper $K$:
%      $$
%      𝒱(⋃_{i ∈ I}𝖆_i) = 𝒱(∑_{i ∈ I}𝖆_i) = ⋂_{i ∈ I}𝒱(𝖆_i)
%      $$
%    \item Sei $K$ ein Körper, $(S_i)_{i ∈ I}$ eine Familie von Teilmengen, $(𝖆_i)_{i ∈ I}$ eine Familie von Idealen und $(𝖗_i)_{i ∈ I}$ eine Familie radikaler Ideal von $K[x_1,\dots,x_n]$.
%      Zeige die folgenden Regeln für $𝒱$ und $ℐ$:
%      \begin{multicols}{2}
%        \begin{enumerate}
%          \item $𝒱(∑𝖆_i) = ⋂𝒱(𝖆_i)$
%          \item $𝒱(I∩J)=𝒱(I)∪𝒱(J)$
%        \end{enumerate}
%      \end{multicols}
%    \item Sei $𝒫(X)$ die durch Inklusion teilweise geordnete Potenzmenge einer Menge $X$.
%      Berechne Supremum und Infimum einer Familie von Teilmengen $(S_i)_{i ∈ I}$ von $X$.
%    \item Sei $\Idl(R)$ die durch Inklusion teilweise geordnete Menge aller Ideale eines kommutativen Ringes $R$.
%      Berechne Supremum und Infimum einer Familie von Idealen $(𝖆_i)_{i ∈ I}$.
%    \item Sei $\Rad(R)$ die durch Inklusion teilweise geordnete Menge der radikalen Ideale eines kommutativen Ringes $R$.
%      Berechne Supremum und Infimum einer Familie von Radikalen $(𝖗_i)_{i ∈ I}$.
%  \end{enumerate}
%\end{exercise}
\end{document}
